\documentclass{book}
\usepackage[utf8]{inputenc}
\usepackage[english]{babel}


\begin{document}
\title{Nameless}
\chapter
Le crépitement des buches dans le foyer se fit nettement plus distinct, le grincement angoissé des ongles de l’aubergiste contre l’attache de son tablier noir de graisse, plus rapide. Les faisceaux de lumières provenant des fenêtres à coté de la porte réchauffaient l’atmosphère pourtant fraiche de la pièce et les particules de poussières semblaient s’employer à former une énième couche de suie sur le sol et le comptoir de l’établissement. Manifestement, la remarque cinglante avait produit l’effet escompté. \\*
Bien qu’il fut dos au reste de la salle, X sentis la choppe de bois traverser les quelques mètres qui le séparait de son agresseur. Courbant légèrement l’échine, l’objet lui passa a quelques centimètres du haut du crane avant de s’écraser sur le premier mur venu. Sautant instinctivement de son tabouret, il l’empoigna de la main gauche et l’envoya voltiger vers son adversaire et plongea vers l’avant. Surpris par la riposte rapide, l’homme corpulent a l’air mauvais n’eut que le temps d’ouvrir de grands yeux avant de recevoir le projectile improvisé dans la poitrine, lui coupant le souffle. Il roula sur lui même, et se redressa maladroitement, rouge de rage. Trop lentement. La lame vint lui sectionner proprement l’index et le majeur qui tentaient d’atteindre la dague dissimulée sous sa chemise, déversant une gerbe écarlate sur le sol. \\*
L'homme se courba et retomba sur ses genoux, tenant de sa main valide sont autre membre et poussant des gémissements gutturaux.
 Les autres hommes présents dans la salle avaient eux aussi sauté de leurs sièges, mais ils étaient désormais figé, blancs comme des linges.
- Quelqu’un d’autre désire se faire délester d’un de ses membres ? envoya X en tournant sur lui même, son épée ensanglantée se balançant au bout de son bras.
N’obtenant aucune réponse, il fit volte face et débarrassa paresseusement sa lame du sang qui la maculait sur les habits de l’homme qu’elle venait de mutiler.
Tout en rengainant son arme, il se dirigea vers la porte principale et l'ouvrit.
Le vent chaud de la fin d’après midi s'engouffra dans l'auberge.
Les battants se refermèrent dans un grincement. Le calme revint.
Les clients encore hébétés, maculés de suie et de poussière reprirent finalement leur esprits et convergèrent vers le blessé. \newline



En sortant du bâtiment, X fut constata que e soleil était bien plus bas qu'il ne l'avait pensé. Il rabattit son capuchon et remonta le petit chemin de gravier fin menant au "Ragondin Grondant" et prit la direction des écuries situées quelques dizaines de mètres plus haut, le long de la route. Il salua d'un mouvement de tête distrait le garçon chargé de surveiller les bêtes, à moitié assoupis les mains croisées sur un ventre déjà proéminent, et alla chercher sa jument dans le box qui lui était réservée puis la fit sortir à l'air libre. Alors qu'il flattait l'encolure de son destrier, des bruits de bottes lui parvinrent, très semblables à ceux qu'il avait émis quelques instants plus tôt accompagnés de conversations ponctués d'injures très recherchées. Ayant tourné la tête, le cavalier désormais en selle aperçu au travers des arbres une file d'une dizaine d'hommes en train de remonter vivement le sentier, armés de se qui lui semblait être les pieds des chaises de l'établissement et de couteaux reluisants dans les derniers rayons du soleil d'automne. Bien que X savait sa maitrise de l'épée amplement suffisante pour tenir tête à deux hommes à la fois, il n'avait nul intérêt à se laisser prendre à parti par un groupe aussi large.
Il tira légèrement sur les rênes en cuir, mit sa monture au trot et profita des branchages pour s'éclipser en longeant le bord de la route. \newline

Après avoir bifurqué plusieurs fois, X décida qu'il était désormais hors de danger et découvrit son crane pour apprécier pleinement la brise sur son visage. Il était censé se reposer après tout. Laissant ses pensées vagabonder, le cavalier traversa plusieurs minutes durant un petit sous-bois ou les champignons colorés jonchaient le sol rocailleux et les pieds d'arbres aux branches penchées d'avoir soutenu le poids des feuilles et des fruits durant les derniers mois. Les bosquets jaunissants qui bordaient le petit sentier remuaient doucement lorsque un couple d'écureuil le traversaient à toute vitesse en batifolant et semblaient les critiquer de toute leur modeste hauteur en se penchant les uns vers les autres. Un peu plus loin le sentier déviait pour contourner une marre d'eau stagnante et verdie ou un groupe de batraciens avaient élu domicile, à en juger par les sons caractéristiques qui en provenaient. La tombé de la nuit ramena X à ses esprits, il s'approchait de la lisière de la forêt et serait bientôt en vue de la ville de V. Il était temps.
Les groupes de voleurs et de bandits appréciaient les d'endroits boisé, là ou se dissimuler était jeux d'enfant pour quiconque savait exploiter les disparités du terrain et les aléas des troncs morts. La lumière faiblissant, les opérations de ce type se multipliaient et bien trop souvent, les marchants itinérants devaient signaler aux autorités locales le lendemain les traces des méfaits accomplis pendant la nuit passée. \newline
X donna quelques coups de talon à sa jument qui jeta en rechignant un regard mauvais à son cavalier puis accéléra imperceptiblement son allure.
Il ne tenait pas à finir avec une pointe de flèche entre les omoplates ou une dague en travers de la gorge.
Alors qu'il faisait la liste de toutes les situations qui pourrait le conduire à se faire occire pour le contenu de sa bourse, il fit finalement irruption dans une plaine inégale, ou quelques moutons peu disciplinés semblaient chuchoter des belles paroles aux touffes d'herbes bien grasses qui parsemaient le sol. \newline

Le jeune homme se rendit compte que son inattention et ses spéculation sur les dangers environnants l'avaient conduit à en oublier de rester à proximité de la route principale, qui menait aux portes de la ville. Il tira sur les rênes et alla s'immiscer dans la file composée de marchants itinérants. L'hiver avait été long et rude. Le printemps était bien installé, laissant le loisir aux carrioles et aux caravanes de fouler les pavés de leur axes de passages habituels.
La ville n'avait pas l'air d'avoir beaucoup changé au cours des deux dernières années. Les murailles, témoignage visible d'un passé lourd de se qui avait du être des batailles héroïques dont le souvenirs s'étaient effacés arboraient toujours leurs larges ouvertures. Celles qui s'étaient trouvés trop proche du sol avaient été rebouchées sans précaution, avec des pierres de toutes tailles et de couleurs disparates, se qui donnait à l'ensemble un style que les habitants qualifiaient de grotesque. Évidemment, aucun d'entre eux n'aurait été prêt à accepter une telle insulte de la part d'un étranger. Ils nourrissaient un amour profond et sincère pour leurs terre et encore plus pour leur histoire dont il existait une nombre de version au moins égale au nombre de personnes en âge de parler dans la cité. De fait, X possédait lui même une version de ce récit, car sa mère lui avait donné la vie a quelques minutes vol d'oiseau seulement de la cité dans laquelle il s’apprêtait à entrer et, même s'il était prêt à admettre que les remparts bariolés et troués avaient une allure peu banale, c'était, selon lui, ce qui leur donnait un charme indéniable. \\*

Il arrivait à quelques dizaines de mètres de l'entrée Est de la ville quand il remarqua que l'accès à la ville était contrôlé activement par un groupe de quatre gardes qui gesticulaient et braillaient des ordres à l'intention des voyageurs.
Etrange. De toutes les fois ou il avait eu l'occasion de traverser V, il ne conservait de souvenirs que d'une paire de sentinelles qui généralement ronflaient bruyamment, confortablement installés sur leurs chaises. Dans le reste des cas, quand ils étaient éveillés, ils s'appliquaient à complimenter sans aucune courtoisie les jeunes demoiselles à l'arrière des charriots, et à se moquer du reste des passants. 
Mais cette fois, quand X passa l'énorme porte en clé de voute, il senti le regard inquisiteur de la sentinelle courir le long de son échine. \\*

"Eh toi, oui toi avec la tignasse !"\\*
"Que puis-je pour vous Monsieur le capitaine de la garde", répondit X avec le ton sarcastique qu'il avait l'habitude d'employer quand on venait lui faire perdre son temps.\\*
"Vous pouvez commencer par descendre de votre cheval, et cesser de faire le mariole, car je connais une cellule dont le gris des murs irait très bien avec votre manteau."\\*
"Loin de moi l'idée de vouloir vous causer le moindre soucis. Je sais a quel point votre activité aux portes de cette ville est exigeante physiquement et mentalement",
lança le jeune homme en arborant son plus beau sourire.\\*

Les yeux de la sentinelle au ventre proéminent se plissèrent légèrement et un silence de quelques secondes s'installa, pendant lequel les deux partis se jaugèrent du regard, l'un fièrement perché en haut de sa monture, l'autre serrant les dents aussi fort qu'il serait le manche de sa lance, pique vers le ciel. Finalement, X sauta à terre, donna quelques caresses à sa jument, et se présenta devant son interlocuteur, le dominant d'une bonne tête. 
Alors commença un examen bien trop long pour se qu'il aurait du être. L'homme bedonnant le fit se retourner sans ménagement et s'appliqua à le fouiller sous toutes les coutures.\\*
L'examen révéla un jeune homme de 24 ans, au teint mate au visage carré composé d'une mâchoire puissante couvert e d'une barde naissante et d'une bouche pincée ainsi que d'une paire d'yeux d'un bleu profond qui produisait sur les membres du sexe opposé un effet particulier, se qui n'était pas pour déplaire à X qui ne manquait pas de tirer parti de ces atouts. Ces cheveux, qui lui avaient valu le doux sobriquet de Tête-de-Noeud de la part de ses deux soeurs, étaient sombres et bouclaient jusqu'au niveau des oreilles donnant l'impression qu'il portait un casque la plupart du temps.\\*
Une fois que le garde eu fini de d'inspecter les poches et l’intérieur de son veston ainsi que la tunique de toile marron terne qu'il couvrait, il jeta un regard intrigué à la longue cicatrice rosée qui partait de l'arrière de la nuque de l'insolent voyageur et courait jusqu'en haut de son épaule droite. Il ne reçu pour réponse qu'un regard noir et un spasme, destiné à réajuster l'emplacement du tissu et ainsi dissimuler au mieux un souvenir douloureux.\\*
"C'est bon tu peux y aller", maugréa le garde.\\*
La déception était visible sur son visage bourru. Probablement de n'avoir rien trouvé qui eu constitué une raison valide de retenir X plus longtemps, histoire de lui passer l'envie de se moquer du capitaine de la garnison de V.\\*
"C'est trop aimable de votre part."\\*
Et le jeune homme se hissa sur le dos de son destrier, le mettant au pas. Il prit le temps, en s'éloignant, de faire un petit geste de la main au capitaine, dont les joues avaient prit une teinte rougeaude au cours des dernières minutes.\\*
X s'engagea dans la première rue à droite après le poste de contrôle. Il laissa ses épaules tomber et souffla un grand coup et donna quelque tapes amicales et chaleureuses à sa jument.\\*
-"Un bel idiot, ce 'capitaine de la garde', il n'a pas prit le temps de vérifier le contenu de mes bagages. Heureusement que j'ai eu la présence d'esprit d'y cacher mon épée. D'habitude je ne m'en serais pas soucié, mais je suis certain que ce benêt  m'aurait épinglé s'il l'avait vue. Tu été parfaite Valha, comme d'habitude."\\*

La jument s'ébroua comme pour remercier son cavalier. C'était un excellent cheval à l'allure singulière et au fort caractère. Sa robe blanche tachetée de tâches noires éparses lui donnait presque un air de vache. Pour autant, elle pouvait rivaliser d'endurance et de vitesse avec les destriers de combat, quand elle ne se laissait pas distraire par les touffes d'herbe alléchantes 
des borts de route ou par un étalon dont les pattes musclées réveillaient en elles des pulsions primitives. Valha avait accompagné X au cours des deux dernières années dans chacune de ses entreprises et leur lien était désormais aussi fort que celui des doigts d'une même main.   \\*
Ils déambulèrent aux travers des ruelles bordées de maisonnettes qui pour la plupart avaient une structure portée par d'énormes tiges de bois. Les ouvertures sur la rue se faisaient rares, et plus particulièrement aux étages les plus proches du sol. Ce genre d'ouvertures étaient généralement une voie d'accès facile pour les cambrioleurs qui délogeaient les carreaux de leur gongs et dépouillait les habitants de l'ensemble de leur bien durant leur sommeil. Les murs et les toits tordus étaient eux constitués de pierres de tailles variables et maintenues ensemble par un mortier grossier fait d'eau et des céréales des récoltes qui étaient de trop mauvaises qualité pour être consommer, réduits en bouilles. Mais le détail architectural le plus singulier était sans l'ombre d'un doute l'agencement de toutes ces modestes bicoques. Elles avaient été bâties les unes après les autres, les unes contre les autres sans aucun soucis d’esthétique ou de symétrie. Par conséquent, les axes de passages suivait des courbes successives et s'éparpillaient s'en prévenir en multitudes de ruelles de plus petite taille comme le feraient les vaisseaux sanguins d'un poumon humain.
Certaines maisons cependant se démarquaient des autres. L'envergure des ailes du bâtiment, bien droit, ne laissait aucun doute sur les sommes astronomiques que la construction avait du demandé aux propriétaires, sans même songer à l'entretien une fois les travaux terminés. Les jardins qui bordaient la maison elle même pullulaient de fleurs colorés au parfum envoutant, d'arbustes biscornus et de petites marres silencieuses et éclairées par des torches de manière à garantir un effet visuel optimal, même de nuit. L'intégrité de ces paradis étaient protégée par de vertigineuses grilles ornementées en leur bout de piques. Des chiens sombres se promenaient également sur le terrain et jouait un rôle dissuasif quand un marcheur imprudent s'approchait d'un peu trop près.
X tourna a droite une dernière fois et déboucha finalement sur une place pavée. Une fontaine trônait en son milieu et était entourée de manière circulaire par des étales vides. 
Le jeune homme s'approcha d'une échoppe qui formait l'angle entre deux rues qui partaient de la place et frappa sur la grosse porte en bois massif. Pas de réponse.
Un léger vent faisait se balancer l'insigne de l'enseigne, qui représentait un pot rempli de laquelle sortait un filet de fumée rougeâtre, dans un grincement lugubre. X frappa à nouveau, avec beaucoup plus d’énergie cette fois.
Enfin, des pas se firent entendre derrière la porte de bois renforcée. Une glissière métallique coulissa et une petite voix annonça : \\*
"L'échoppe est fermé, revenez demain !" \\*
"Viktor, c'est moi, X, je viens voir ton maitre." \\*
La porte s'ouvrit avec peine après quelques secondes, et pour cause, l'enfant qui se tenait dans l’embrasure de la porte n'avait qu'une dizaine d'années tout à plus. \\*
"Il est à l'arrière, harnaches ta jument dans la cour" dit il. Une fois la mangeoire du destrier remplie généreusement, le jeune garçon saisit le visiteur nocturne par la main (sa poigne était étonnement ferme  pour un enfant) et ensemble ils traversèrent le magasin, dont les murs était couvert d'étagère bien organisées, garnies de bocaux remplis de feuille, racines et décoctions à l'aspect peu ragoutant étiquetées et triées par ordre alphabétique. Ils passèrent derrière le comptoir et la tenture qui faisait la jonction entre le magasin et l'étage qui le surplombait.
Bien qu'il avait déjà visité l'endroit à de nombreuses reprises, X 	fut une fois de plus surpris 
par l'étrangeté de la décoration des lieux, qui tranchait fortement avec le style de la boutique. La s'entassaient coffres et sacs de toiles bourrés dont certains, percés, déversaient nonchalamment leur précieux contenu sur le plancher. Les murs  dénudés et dépourvu de sources de lumière prenaient des angles inattendus se qui rendait la navigation difficile. Après une volée d'escaliers qui montaient en colimaçon séparant l'étage supérieur d'un couloir qui donnait sur la cour intérieure, ils entrèrent dans une pièce carrée: laboratoire du maitre des lieux. Plusieurs longues tables juxtaposées formaient un atelier qui faisaient le tour de la pièce et n'était interrompu que par la présence d'une autre porte symétriquement opposée à celle par laquelle ils étaient entré. Des odeurs de combustion provenaient des alambics et des pots ouverts se qui rendait l'air nauséabond et irrespirable. A chaque pas, un petit nuage de particules s'élevait du sol. Combiné à la faible lueur que procurait le bout de chandelle que tenait l'enfant de ses doigts boudinés, cet épais nuage flottant doucement, l'incroyable attirail et le silence oppressant donnaient au lieux un ton quelques peu mystique et, bien qu'il savait que se n'était pas le cas, X ne se serait pas étonné que l'endroit soit fréquenté par un groupe de manipulateurs de l'Ether. C'était dans un endroit similaire qu'il aurait sans soucis imaginé un groupe de vieillard, aigris par les années et par la rudesse que requérait la pratique leur art obscur, passer le peu de temps qu'ils leur restaient.
\\* "Arrêtes donc de rêvasser, les vapeurs vont te donner une sacrée migraine si tu reste trop longtemps dans l'atelier". \\* 
Sur ce, Viktor repris sont avancée rapide et fit passer le curieux dans le salon en l'entrainant par la manche. Un feu crépitait joyeusement dans l'âtre. Deux fauteuils très accueillants l'encadrait, dont l'un était déjà occupé par une longue silhouette avachie.
L'homme se leva péniblement à leur entrée et vint à leur rencontre.
Le poids de l'age et les longues années l’expérience de l'herboriste se distinguaient sur son visage carré (tout à fait à l'image de son atelier, se dit X à lui même) aussi facilement que deux topazes au milieu d'un tas de charbon. Il avait des traits bourrus qui tranchaient avec un nez aquilin, des sourcils fournis au point de ne presque plus laisser de place à deux petits yeux pétillants. Une touffe blanche inégale garnissait son crane et privait la gravité de ses droits en s'éparpillant en une forme inqualifiable qui aurait fait rougir n'importe quel explosif. 
Le chemisier à manches longues et aux couleurs vives qu'il avait enfilé pour la journée indiquait implicitement sons appartenance à la tranche la plus aisée de ceux dont le labeur constituait la majeur partie de la journée. Les artisans dont le travail était d'une qualité suffisante pour attirer les acheteurs dont les bourses étaient les plus remplies pouvaient s'offrir le luxe de parer d'excentricités de ce genre les tenues qui les différenciaient du bas peuple. Leurs confection demandaient une extraction délicate de fleurs et de fruits qui ne poussait pas dans les régions avoisinantes. Certains racontaient même que de rares coloris étaient d'origine animale, mais les secrets de leur travail étaient si farouchement gardés par les quelques personnes qui les possédaient, qu'il était impossible de démêler la vérité des vulgaires spéculations de comptoir du premier ivrogne venu. Malgré son apparence singulière, X ne doutait pas que le vieillard avait conservé sa vivacité d'esprit, et plus encore, que le tranchant de sa langue n'avait pu que s'affuter au fil de ses deux dernières années, se qui le fit frémir de tout son être.  L'homme leva ces deux bras en l'air, laissant apparaitre les longues baguettes décharnées au bout de chacune de ses paumes. \\*
"X! ", déclara il enfin. \\*
"Orto ! Tu as l'air en très bonne forme" répondit X en lui donnant une tape énergique qui fit chanceler le maitre de maison.\\*
"Je ne rajeunis pas, mais je n'ai pas encore dis moi dernier mot. Je dois dire que je suis un peu surpris de te t'entendre toquer à ma porte, qui plus est à une heure si tardive. Mais ne te méprends pas, je suis bien heureux de te savoir de retour"\\*
"Désolé papi, je n'ai pas eu l'occasion de m'annoncer. Tu me pardonnera rapidement de mon manque de manières quand je t'aurai rapporté des nouvelles de la frontière Nord, je sais que tu les vends bien plus chères que tes potions douteuses"\\*
"Manifestement, tu connais mon métier bien mieux que moi." Les yeux du vieillard s'étaient mis à étinceler d'un éclat nouveau, comme si sa jeunesse et sa férocité lui étaient revenus l'espace de quelques secondes, à la perspective de futurs profits.\\*
"Ce sera avec plaisir, mais je dois d'abord m'assurer de ne pas m'endormir au cours de ton récit. Veux tu une tasse de thé ?"\\*
"Sans façon, je suis bien trop jeune pour mourir au beau milieu de la nuit avec un filet de bave au coin de la bouche".\\*
Un grand sourire se dessina sur le visage du jeune homme et le dénommé Orto ignora la remarque, se retourna et s'en fit préparer son remontant en psalmodiant des injures qui tournaient autour du manque d'éducation de ces "tas de muscle de l'Ordre".
Quelques instants plus tard, ils s'installaient tout deux confortablement face à face dans les fauteuils du salon. Viktor, qui était resté silencieux, s'était également installé: à même le sol, visiblement déterminé à ne pas manquer une miette de la discussion. Sans nul doute pouvait il aussi lui tirer quelques piécettes de ses histoire s'il les transmettaient à ses propres clients. Tous les jeunes garçons qui n'avaient pas eu la chance d'être accueillis et formés survivaient de travaux de rue ou d'escroqueries juvéniles. Informer et surtout désinformer en faisaient parti et étaient même très sérieusement considéré par les marchands itinérant comme un facteur étroitement lié à la réduction du nombre de concurrents et par conséquent, la multiplication de leur profits. Ainsi, lors du passage de certains groupes de voyageurs sous les grandes portes bariolées, la ville se transformait en formidable fourmilière où les ruelles étaient parcourues de long en large et dans tous les sens possible par des garnements qui courraient aussi vite que leurs courtes jambes pouvaient les porter. \\*
"Bien, je t'écoute" déclara Orto en prenant une gorgée de sa tasse fumante.\\*
"Tu te souviens probablement de mon départ, il y a deux ans, pour l'Ordre. A l'époque je me  l'imaginais comme une grande famille de guerriers dont les membres avaient voués leur vie à la protection de C (note à retirer : petit Etat dans lequel se déroule l'action), notamment des raids de Slyghs qui descendent des Fer-Dents chaque année à l'approche du printemps." \\*
"Je m'en souviens très bien, X. " Répondis son interlocuteur en hochant la tête de haut en bas.\\*
"Durant les 8 premiers mois, j'ai été intégré à une cohorte de bleus et l'entrainement fût long et rude. Des heures et des heures d'enseignement sur le maniement de l'épée, de la hache et des tourne-lame. Une unité et des liens d'amitiés se sont vite crées et nous nous sommes livré de nombreux combats factices en attendant patiemment l'heure de gloire de notre première bataille. J'ai pensé beaucoup de blessures une fois l'entrainement quotidien achevé et je croyais niaisement qu'après mon premier passage sur le champ de bataille, je ferai la même chose et tout recommencerait. Je me trompais ! Les premières chaleurs du printemps son arrivées et avec elles les colonnes de Slygh. As tu déjà vu un Slygh Orto ?" \\*
"Je ne peut pas dire que j'ai eu cette chance" murmura l'herboriste avec sarcasme. \\*
"A première vu, on pourrait les croire humains. Mais une fois que l'un d'entre eux fixe sur toi ses deux yeux teintés du désir de t'arracher les membres, de se repaitre ta chair, de ton sang, il devient impossible de s'y méprendre". Le jeune homme avait inconsciemment crispé les muscles de ses bras, et tout son corps  suintait des ressentiments qu'il nourrissait à l'égard des créatures.
L'air était de la pièce était immobile et électrique. Après quelques secondes de silence, X releva la tête, une lueur de feu animait son regard. Il repris son discours.\\*
"Le meilleur moyen de les décrire serait de visualiser un bouc à la silhouette humaine sur ses deux pattes arrières. Bien sur, il ne possèdent pas la quantité de laine de leurs homologues qui broutent notre herbe, mais il ont cependant un épais duvet qui les recouvre presque intégralement, qui amorti les coups et protège leurs organes vitaux. Leur tête en est privé mais les deux cornes courbées qui s'y trouvent sont aussi tranchantes que des poignards. Au moindre contact, elles vous ouvre la peau et réduisent en bouillie tout se qui se trouve en dessous".
La dessus, il tira légèrement sur le tissu qui recouvrait son épaule, découvrant ainsi la longue cicatrice qui lui bardait l’omoplate.\\*
"Pire encore : leur hurlements strident. Il rugissent à chaque nouvelle victime et leur férocité en est ainsi renforcée. Nous avons malgré tout réussis à les repousser et à limiter les pillages". Le compteur examina son épaule meurtrie l'espace d'un instant puis releva la tête doucement. \\*

"Viktor, voudrais tu bien aller chercher de quoi alimenter ce pauvre feu, à ce train dans cinq minutes, on pourra me décrocher le bras rien qu'en tirant dessus".\\*

En effet les braises mourraient dans l'âtre. Viktor se leva à contre cœur et, craignant de manquer la meilleure partie de la discutions s'en fit chercher en de quoi alimenter le foyer aussi rapidement que possible. Les deux hommes le regardèrent s'éloigner. A l'instant même ou le l'enfant eu traversé le seuil, X attrapa l'herboriste par la manche et l'approcha de lui, si bien qu'il n'y eu plus que quelques centimètres qui les sépara.\\*

"Cependant, cette année, lorsqu'ils sont descendus de leur montagnes, ces maudits homme-boucs n'était pas armés comme à leur habitude, avec des grossières massues ou gourdins improvisés.
C'est avec de vraies haches d'acier et de lances pointues qui nous venu. Et c'est avec ces mêmes instruments qu'ils ont décimé l’intégralité de la cohorte." \\*

"Je crois comprendre se que tu essayes de dire, X. Mais dans ce cas, que fais tu ici ?"\\*

"Orto, écoutes moi, les Slygh sont incroyablement difficiles à abattre et terriblement dangereux, mais ils ne sont pas très malins. Tout cet attirail ne leur appartenait pas ! Qui plus est, ces armes étaient taillées pour des mains humaines. Après notre dernière bataille, il y a de cela trois semaines désormais, les quelques hommes qui avaient survécus furent renvoyés chez eux avec pour consigne de s'y reposer jusqu’à nouvel ordre. Et me voila qui frappe à ta porte. Or, je sais que l'Ordre ne se priverait pas d'un seul hommes après les pertes subies. Quelque chose ne tourne pas rond ! J'ai comme l'impression que l'Ordre essayes de se débarrasser des témoins restants de la bataille. Nous nous connaissons depuis fort longtemps désormais Orto, et je sais que cette dernière information te permettrait de renouveler tout ton attirail farfelu, mais personne ne doit être au courant, et surtout pas ton apprenti." Chuchota le jeune guerrier survolté.\\*

"Tu as ma parole." lui répondit le vieillard. \\*
"C'était se que j'étais venu te demander, à vrai dire. Mais par pitié, discrètement !"\\*
"Évidemment."\\*
Quelques secondes plus tard le garçon revint avec son lourd butin entre les bras et retrouva son maitre et son hôte débattant de la dernière extravagance vestimentaire très en vogue pour les robes féminines en riant aux éclats. Cette dernière, qui consistait en une ouverture dans la bas du dos destinée à découvrir la partie supérieure des fesses de manière similaire à un décolleté 'classique', n'avait évidemment aucun intérêt aux yeux de Viktor (littéralement). \\*
"X, je veux en savoir plus sur les hordes de Slygh !".\\*
"Orto et moi sommes fatigués, nous allons nous coucher. Une autre fois peu être."\\*
"Mais vous n'avez pas l'air fatigués du tout !" rétorqua l'apprenti en bougonnant.
"Tu n'a pas encore mon âge Viktor" dit le maitre à son élève dans un rictus malin, coupant ainsi court à toute forme d'argumentation".\\*  	
Vexé de s'être manifestement fait évincé, le jeune garçon, toujours accompagné de ses buches, se dirigea en trainant des pieds vers la porte pars laquelle il venait d'entrer, suivi de près par les deux bavards.



 


\end{document}